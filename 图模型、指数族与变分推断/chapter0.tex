\chapter*{摘 \quad 要}
\addcontentsline{toc}{chapter}{\textbf{摘 \quad 要}}

概率图模型(Probabilistic graphical models, PGMs)这一形式化工具为捕获随机变量之间复杂的依赖性从而建立大规模多变量统计模型提供了一个统一的框架。
在统计、计算机与数学的众多领域中,图模型已经成为了一个研究的热点,包括生物信息、通讯理论、统计物理、组合优化、信号和图像处理、信息检索以及统计机器学习等。
许多从特例中发现的问题,例如边缘分布与众数的计算,已经在一般条件下得到了很好的研究。
我们使用指数族分布(Exponential families)进行概率表示,利用指数族的累积函数(Cumulant function)和熵(Entropy)的共轭对偶性(Conjugate duality),提出了一套可用于计算似然(Likelihoods)、边缘概率(Marginal probabilities)和最概然配置(Most probable configurations)的通用变分表示方法(Variational representations)。
我们对一大群算法变体进行了描述,包括和积算法(Sum-product)、聚类变分法(Cluster variational methods)、期望传播算法(Expectation-propagation)、平均场方法(Mean field methods)、最大积算法(Max-product)、线性规划松弛法(Linear programming relaxation)以及二次规划松弛法(Conic programming relaxation),这些算法都可以理解为是变分表示的精确或近似形式。
变分方法为大规模统计模型的近似推断提供了与马尔可夫链蒙特卡洛法(Markov Chain Monte Carlo, MCMC)完全互补的另一项选择。
