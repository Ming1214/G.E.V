\chapter{引言}

图模型将图论和概率论结合在一起,构成了一个用于多变量统计建模的强大形式框架。
在包括生物信息、语音处理、图像处理和控制论等多个应用领域内,统计模型早已习惯于使用图来进行形式化表达,而计算如似然和得分函数等基本统计量的算法也常被表示为在这些图上的递归操作,例如物种进化树、遗传图谱、隐马尔可夫模型、马尔可夫随机场以及卡尔曼滤波等。
这些想法都能使用图模型的形式进行理解、统一和泛化。
图模型不仅提供了一种很自然的工具来阐述这些经典结构的变体,还可以用于探索全新的统计模型族。
于是在涉及到需要研究大量有交互作用的变量的领域,图模型的使用频率在显著地升高。

在许多面向计算的领域里,图论扮演了十分重要的一个角色,包括组合优化、统计物理和经济学等。
除了可以用来形式化模型之外,图论也在评价计算的复杂度和灵活度中起到了基础性的作用。
特别地,一个算法的运行时长或者是误差界的量级也经常能够表示为某个图的结构特征。
图论起到的这种作用在图模型中也同样存在。
例如我们之后会讨论的名为“联合树算法”(Junction tree algorithm)——一种上文提到的在图上的递归算法的一般化版本——的计算复杂度就能够用相互作用的变量所构成的图的理论度量来表示。
对于某些特殊的稀疏图,联合树算法提供了一套系统的方案可以用来计算图模型上的似然和其他统计量。

不幸的是,许多实际的图模型并不是稀疏的,所以联合树算法不能提供可行的计算方案。
MCMC 框架是一种很受欢迎的方法,可以用来解决上面那种情况,大量的文献都在图模型上应用了 MCMC 方法。
然而我们的目标有些不一样:我们提出另一种基于变分方法的统计推断计算方法论。
这项技术不仅提供了 MCMC 的替代品,还具有在图模型框架之外的应用。
然而我们将会看到变分方法能够很自然地运用到图模型中去,因为变分方法和图的结构属性有着密切的联系。

“变分”本身是一个概括性的词汇,指的是把问题用优化的形式进行表达和求解的各种数学工具。
一般的想法是把感兴趣的量表示为一个优化问题的解。
优化问题可以通过不同的方式放宽约束条件,可以对优化目标函数进行近似处理,也可以对可行域进行近似处理。
这些放宽约束的手段提供了一种原问题的近似方法。

MCMC 和变分方法都来自于统计物理。
MCMC 方法在进入统计领域之前就已经在统计物理中取得了很好的应用效果,这也促进了统计领域对这一方法的理论研究。
事实上,为特定统计问题所设计的 MCMC 方法的发展为此类方法在统计领域的广泛应用起到了重要作用。
变分方法可以借鉴这种发展方式,在我们看来,将变分方法应用到统计领域最有希望的途径是利用好变分分析和指数族分布之间的联系上。
这是因为处于指数族统计理论核心位置的凸性概念对变分松弛的设计有很大的作用。
此外,这些变分松弛法在图模型的结构上可以产出很有意思的算法,这些算法同样以在图上递归的形式实现。

我们对三个相关的主题呈现了一个完整的故事。
我们在第二章中讨论图模型,在提供一般的数学框架的同时也针对几个特例进行讲解。
所有这些例子以及当前图模型的大多数应用都涉及到指数族分布。
我们在第三章中讨论指数族,着重于指数族和凸分析之间的数学联系,在此基础上给出变分方法的发展脉络。
需要特别突出说明的是指数族的某种共轭对偶关系。
在共轭对偶关系的基础上,我们提出了用于计算指数族似然和边缘概率方法的一般变分表示。
随后的部分将致力于探索这个变分原理的各种实例,包括精确和近似形式,从中可以得到精确或近似计算边缘概率的各种算法。
我们在第四章中讨论贝特近似(Bethe approximation)与和积算法之间的联系,包括在树结构上的精确形式和在带环图结构上的近似形式。
我们也会讨论类贝特近似(Bethe-like approximation)与其他算法之间的联系,包括广义和积算法(Generalized sum-product)、期望传播算法以及相关的矩匹配(Moment-matching)方法。
我们在第五章中讨论平均场方法,这是一种不同于精确变分原理的近似方法,可以生成似然下界。
我们在第六章中讨论变分方法在参数估计中所起到的作用,包括全观测与部分观测两种情况的例子,也包括频率派和贝叶斯派两种不同的视角。
贝特型(Bethe-type)和平均场方法都是基于非凸优化问题的,这类问题经常有多重解。
相较之下我们在第七章中讨论基于精确变分原理凸松弛的变分方法,这类方法大多数都能保证对数似然存在上界。
我们在第八章中讨论众数(Mode)的计算问题,特别强调离散随机变量的情况,这种情况下计算众数需要解决一个整数规划问题(Integer programming problem)。
我们发展了(重新加权的)最大积算法与层次线性规划松弛之间的联系。
我们在第九章中讨论更为一般的二次规划松弛,并且从矩矩阵(Moment matrices)的半定约束(Semi-definite constraints)的视角来进行理解。
我们在第十章进行总结。

这篇文章的范围限于以下情形:给定表示为图模型的一个概率分布,我们主要关心计算边缘概率(包括似然)和众数的问题。
我们习惯上将这种计算任务称为“概率推断”或者简称“推断”。
正如 MCMC 方法的介绍一样,我们关注的主要是贝叶斯统计中的应用。
虽然贝叶斯统计在我们的介绍中占主要部分,但是这些方法可以运用到整个统计范围中去,包括频率和贝叶斯两种范式,我们在这篇文章中指出了一些交叉性的应用。


